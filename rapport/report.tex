%        File: Report.tex
\documentclass[a4paper]{article}

% Packages
\usepackage[paper=a4paper,margin=1in]{geometry}
\usepackage[utf8]{inputenc}
\usepackage[francais]{babel}
\usepackage[T1]{fontenc}
\usepackage{graphicx}
\usepackage{listings}
\usepackage{pbox}
\usepackage{changepage}

\begin{document}

%% TITLING

% Top right, name
\null\hfill\begin{tabular}[t]{l@{}}
  \textbf{Alexandre Lewkowicz} \\
  \textbf{Maxence de Bergeyck} \\
  \textit{SCIA}
\end{tabular}

% Space before title
\vspace{1cm}

% Title
\noindent\makebox[\textwidth][c]{
  \Large\bfseries PLAM, projet}

% Space avter title
\vspace{1cm}

%% CONTENTS

\subsection*{Partie I - Modélisation}

\textbf{Soient les constantes données suivantes:}

\begin{description}
    \item[$M$] \hfill \\
        Le nombre d'entrepots
    \item[$N$] \hfill \\
        Le nombre de sites disponibles
    \item[$C_{i}$] \hfill \\
        La capacité de l'entrepot $i$
    \item[$D_{i, j}$] \hfill \\
        La distance entre les entrepots $i$ et $j$
\end{description}

\textbf{Soient les variables résultat suivantes, qu'on définit:}

\begin{description}
    \item[$b_{i}$] \hfill \\
        Valeur booléenne indiquant si le site $i$ a été choisi.
\end{description}

\textbf{On définit la fonction objectif (à maximiser) suivante:}

$$
Z = \sum\limits_{i = 1}^{N}b_{i} * C_{i}
$$

On somme la capacité des entrepots activés.

\textbf{Ainsi que les contraintes suivantes:}

\begin{description}
    \item[$\sum\limits_{i = 1}^{N}b_{i} = M$] \hfill \\
        La somme des booléens doit être égale au nombre de sites requis
    \item[$\forall i \in \lbrack1, N\rbrack\  \forall j \in \lbrack1, N\rbrack\quad  b_{i} * b_{j} * D_{i, j} <= 50km$] \hfill \\
        Les couples dont les deux sites sont présents ne doivent pas être
        éloignés de plus de 50 km.
\end{description}

\textbf{Touche finale:}

Le problème de cette modélisation est que la contrainte de distance est \textbf{quadratique}.
On cherche donc une astuce pour exprimer le problème différemment.

On introduit des variables intermédiaires $\omega_{i, j}$ représentant le ET logique
entre les variables $b_{i}$ et $b_{j}$. Afin que ces variables aient le bon comportement
on applique les contraintes suivantes:

\begin{itemize}
    \item[] $\omega_{i, j} <= b_{i}$
    \item[] $\omega_{i, j} <= b_{j}$
\end{itemize}

qui sont caractéristiques du ET logique. Cependant, cela ne force aucunement les
$\omega$ à prendre la valeur $1$. On ajoute donc les $\omega$ à la fonction
objectif avec un gros coefficient $K$, et on corrige le resultat avec le terme constant de la fonction objectif car
on sait combien de couples $\omega_{i, j}$ doivent être activés ($\mathcal{C}_{N}^{M}$).
On obtiens la nouvelle fonction objectif (à maximiser) suivante:

$$
Z = \left(\sum\limits_{i = 1}^{N}b_{i} * C_{i}\right)
+ \left(\sum\limits_{i = 1}^{N}\sum\limits_{j = i + 1}^{N} \omega_{i, j} * K\right)
- \left(\mathcal{C}_{N}^{M} * K\right)
$$

\subsection*{Partie II - Modélisation, le retour de la revanche}

Je n'ai pas vraiment trouvé une deuxième modélisation du problème.

\end{document}
